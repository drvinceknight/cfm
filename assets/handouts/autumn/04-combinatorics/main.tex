\documentclass{article}

\usepackage{amsmath}
\usepackage[margin=1.5cm, includefoot, footskip=30pt]{geometry}
\usepackage[parfill]{parskip}
\usepackage[black]{merriweather} %% Option 'black' gives heavier bold face 
\usepackage[T1]{fontenc}
\usepackage{hyperref}
\usepackage{minted}
\usepackage{multicol}

\pagenumbering{gobble}

\title{Computing for Mathematics: Handout 4}
\date{}

\begin{document}

\maketitle


This handout contains a summary of the topics covered and an activity to
carry out prior or during your lab session.

At the end of the handout is a specific coursework like exercise.

For further practice you can do the exercises available at 
\href{https://vknight.org/pfm/tools-for-mathematics/05-combinations-permutations/introduction/main.html}{the
combinatorics chapter of Python for Mathematics}.

\section{Summary}\label{summary}
\hrule


The purpose of this handout is to cover combinatorics which
corresponds to the
\href{https://vknight.org/pfm/tools-for-mathematics/05-combinations-permutations/introduction/main.html}{combinatorics
chapter of Python for Mathematics}.

The topics covered are:

\begin{itemize}
\item
  Generating and counting permutations and combinations of elements.
\item
  Directly computing binomial coefficients. 
\end{itemize}


\section{Activity}\label{activity}
\hrule

We will be tackling the problem from the
\href{https://vknight.org/pfm/tools-for-mathematics/05-combinations-permutations/tutorial/main.html}{tutorial
of the combinatorics chapter of Python for Mathematics}.

The digits 1, 2, 3, 4 and 5 are arranged in random order, to form a five-digit number.

\begin{enumerate}
    \item How many different five-digit numbers can be formed?
    \item How many different five-digit numbers are:
        \begin{enumerate}
            \item Odd
            \item Less than 23000
        \end{enumerate}
\end{enumerate}

There are instructions for how to do all of this is in the
\href{https://vknight.org/pfm/tools-for-mathematics/05-combinations-permutations/how/main.html}{combinatorics chapter of Python for Mathematics}.


\begin{enumerate}
    \item Create a variable \mintinline{python}{digits} which has value the
        collection of integers 1, 2, 3, 4 and 5.
\item
    Use the \mintinline{python}{itertools.permutations} tool to create the
        variable \mintinline{python}{permutations} which has value the
        permutations of the integers 1, 2, 3, 4 and 5. 
    \item Convert the \mintinline{python}{permutations} variable to a
        \mintinline{python}{tuple} and use the \mintinline{python}{len} command
        to count the number of different permutations that exist.
    \item Use the \mintinline{python}{sum} command to calculate:
        \[\sum_{\pi \in \Pi}\pi_5\mod 2\] where $\Pi$ is the collection of all
        permutations. 
    \item Use the \mintinline{python}{sum} command to calculate:

        \[\sum_{\pi \in \Pi \text{ if }\pi_1 10 ^ 4 + \pi_2 10 ^ 3 + \pi_3 10 ^ 2 + \pi_4 10 + \pi_5 \leq 23000} 1 \] 
        where $\Pi$ is the collection of all
        permutations. 
\end{enumerate}


\section{Coursework like exercise}
\hrule


\begin{enumerate}
    \item Create a variable \mintinline{python}{number_of_permutations} that gives the 
number of permutations of size 4 of:

\mintinline{python}{pets = ("cat", "dog", "fish", "lizard", "hamster")}

Do this by generating and counting them.
\item Create a variable \mintinline{python}{direct_number_of_permutations} that gives 
    the number of permutations of \mintinline{python}{pets} of size 4 by 
        direct computation.
\end{enumerate}

\section{Summary examples}
\hrule

\begin{multicols}{2}
    Create the collection: \((A, A, B)\).

        \begin{minted}[frame=single]{python}
collection = ("A", "A", "B")
    \end{minted}

        Obtain the 2nd element in the collection \((A, A, B)\).

        \begin{minted}[frame=single]{python}
collection = ("A", "A", "B")
collection[1]
        \end{minted}

        Check a boolean condition like if \("C"\in (A, A, B)\).

        \begin{minted}[frame=single]{python}
collection = ("A", "A", "B")
"C" in collection
        \end{minted}

        Create the collection of integers from 1 to 11:

        \begin{minted}[frame=single]{python}
range(1, 12)
        \end{minted}

        Create the permutations of the collection \((A, A, B)\) of size 2.

        \begin{minted}[frame=single]{python}
import itertools
collection = ("A", "A", "B")
itertools.permutations(collection, r=2)
        \end{minted}

        Create the combinations of the collection \((A, A, B)\) of size 2.

        \begin{minted}[frame=single]{python}
import itertools
collection = ("A", "A", "B")
itertools.combinations(collection, r=2)
        \end{minted}

        Calculate \(\sum_{i=1\text{ if }i\text{ even}}^{20}i ^ 2\)

        \begin{minted}[frame=single]{python}
sum(i ** 2 for i in range(1, 21) if (i % 2 == 0))
        \end{minted}

        Obtain \(5!\):

        \begin{minted}[frame=single]{python}
import math
math.factorial(5)
        \end{minted}

        Obtain \({5 \choose 2}\):

        \begin{minted}[frame=single]{python}
import scipy.special
scipy.special.comb(5, 2)
        \end{minted}

        Obtain \(^5 P_2\):

        \begin{minted}[frame=single]{python}
import scipy.special
scipy.special.perm(5, 2)
        \end{minted}


\end{multicols}

\end{document}
