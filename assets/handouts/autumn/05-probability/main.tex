\documentclass{article}

\usepackage{amsmath}
\usepackage[margin=1.5cm, includefoot, footskip=30pt]{geometry}
\usepackage[parfill]{parskip}
\usepackage[black]{merriweather} %% Option 'black' gives heavier bold face 
\usepackage[T1]{fontenc}
\usepackage{hyperref}
\usepackage{minted}
\usepackage{multicol}

\pagenumbering{gobble}

\title{Computing for Mathematics: Handout 5}
\date{}

\begin{document}

\maketitle


This handout contains a summary of the topics covered and an activity to
carry out prior or during your lab session.

At the end of the handout is a specific coursework like exercise.

For further practice you can do the exercises available at 
\href{https://vknight.org/pfm/tools-for-mathematics/06-probability/introduction/main.html}{the
probability chapter of Python for Mathematics}.

\section{Summary}\label{summary}
\hrule


The purpose of this handout is to cover probability which
corresponds to the
\href{https://vknight.org/pfm/tools-for-mathematics/06-probability/introduction/main.html}{probability
chapter of Python for Mathematics}.

The topics covered are:

\begin{itemize}
\item
  Generating random numbers
\item
  Randomly sample from a given collection of items.
\item 
  Write python functions to be able to repeat experiments.
\end{itemize}


\section{Activity}\label{activity}
\hrule

We will be tackling the problem from the
\href{https://vknight.org/pfm/tools-for-mathematics/06-probability/tutorial/main.html}{tutorial
of the probability chapter of Python for Mathematics}.

An experiment consists of selecting a token from a bag and spinning a coin. The
bag contains 5 red tokens and 7 blue tokens. A token is selected at random from
the bag, its colour is noted and then the token is returned to the bag.

When a red token is selected, a biased coin with probability \(\frac{2}{3}\) of landing heads is
spun.

When a blue token is selected a fair coin is spun.

\begin{enumerate}
    \item What is the probability of picking a red token?
    \item What is the probability of obtaining Heads?
    \item If a heads is obtained, what is the probability of having selected a red token.
\end{enumerate}

There are instructions for how to do all of this is in the
\href{https://vknight.org/pfm/tools-for-mathematics/06-probability/how/main.html}{probability chapter of Python for Mathematics}.


\begin{enumerate}
    \item Create a variable \mintinline{python}{bag} which has value the
        list with 5 copies of the string \mintinline{python}{"Red"} and 7
        copies of the string \mintinline{python}{"Blue"}.
    \item Write a python function \mintinline{python}{pick_a_token} which can be
        used to randomly choose an element from any given container.
    \item Create a variable \mintinline{python}{samples} which has value a list
        with 10,000 random choices from \mintinline{python}{bag}.
    \item Count how many of the elements of \mintinline{python}{samples} are
        \mintinline{python}{"Red"} and use this to approximate the probability of
        picking a red token.
    \item Write a python function \mintinline{python}{sample_experiment} which
        carries out a single instance of the experiment described in the
        question. This function should return both the color of the token and
        the face of the coin.
    \item Create a variable \mintinline{python}{samples} which has value a list
        with 10,000 outcomes of the experiment.
    \item Use \mintinline{python}{samples} to approximate the probability of picking
        Heads.
    \item Use \mintinline{python}{samples} to approximate the probability of picking
        a red token given that Heads is selected.
\end{enumerate}


\section{Summary examples}
\hrule

\begin{multicols}{2}
    Create the collection \((A, A, B)\) as a list.
 
        \begin{minted}[frame=single]{python}
collection = ["A", "A", "B"]
    \end{minted}

        Modify the 2nd element in the collection \((A, A, B)\).

        \begin{minted}[frame=single]{python}
collection = ["A", "A", "B"]
collection[1] = "C"
        \end{minted}

        Include a new element in a collection:

        \begin{minted}[frame=single]{python}
collection = ["A", "A", "B"]
collection.append("C")
        \end{minted}

        Define and call $f(x)=x^2 + 1$:

        \begin{minted}[frame=single]{python}
def square(x):
    """
    Returns x ^ 2 + 1
    """
    return x ** 2 + 1

square(5)
        \end{minted}

        \vspace{2cm}
        Running conditional code:

        \begin{minted}[frame=single]{python}
def absolute_value(x):
    """
    A function that returns -x if x is 
    negative and returns x otherwise
    """
    if x < 0:
        return - x
    return x
        \end{minted}

        Create a list with the numbers $(0 ^ 2, 1 ^ 2, 2 ^ 2, 3 ^ 2, 4 ^ 2)$:

        \begin{minted}[frame=single]{python}
[x ** 2 for x in range(5)]
        \end{minted}

        Sample from the numbers $(0, 1, 2)$

        \begin{minted}[frame=single]{python}
import random

random.seed(0)
random.choice(range(3))
        \end{minted}

        Sample a random number between 0 and 1:

        \begin{minted}[frame=single]{python}
import random
random.seed(0)
random.random()
        \end{minted}
\end{multicols}

\end{document}
