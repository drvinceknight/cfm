\documentclass{article}

\usepackage{amsmath}
\usepackage[margin=1.5cm, includefoot, footskip=30pt]{geometry}
\usepackage[parfill]{parskip}
\usepackage[black]{merriweather} %% Option 'black' gives heavier bold face 
\usepackage[T1]{fontenc}
\usepackage{hyperref}
\usepackage{minted}
\usepackage{multicol}

\pagenumbering{gobble}

\title{Computing for Mathematics: Handout 6}
\date{}

\begin{document}

\maketitle


This handout contains a summary of the topics covered and an activity to
carry out prior or during your lab session.

At the end of the handout is a specific coursework like exercise.

For further practice you can do the exercises available at 
\href{https://vknight.org/pfm/tools-for-mathematics/04-matrices/introduction/main.html}{the
matrices chapter of Python for Mathematics}.

\section{Summary}\label{summary}
\hrule


The purpose of this handout is to cover matrices which
corresponds to the
\href{https://vknight.org/pfm/tools-for-mathematics/04-matrices/introduction/main.html}{probability
chapter of Python for Mathematics}.

The topics covered are:

\begin{itemize}
\item
  Creating matrices.
\item
  Manipulating matrices.
\item 
  Solving a system of linear equations using matrices.
\end{itemize}


\section{Activity}\label{activity}
\hrule

We will be tackling the problem from the
\href{https://vknight.org/pfm/tools-for-mathematics/04-matrices/tutorial/main.html}{tutorial
of the matrices chapter of Python for Mathematics}.

The matrix $A$ is given by $A=\begin{pmatrix}a & 1 & 1\\ 1 & a & 1\\ 1 & 1 & 2\end{pmatrix}$.

1. Find the determinant of $A$
2. Hence find the values of $a$ for which $A$ is singular.
3. For the following values of $a$, when possible obtain $A ^ {- 1}$ and confirm
   the result by computing $AA^{-1}$:
    1. $a = 0$;
    2. $a = 1$;
    3. $a = 2$;
    4. $a = 3$.

There are instructions for how to do all of this is in the
\href{https://vknight.org/pfm/tools-for-mathematics/04-matrices/how/main.html}{probability chapter of Python for Mathematics}.


\begin{enumerate}
    \item Create a variable \mintinline{python}{A} which has value the
        matrix \(A\).
    \item Create a variable \mintinline{python}{determinant} which has value the 
        determinant of \(A\).
    \item Find the values of $a$ for which the determinant of \(A\) is 0. This
        corresponds to the values for which \(A\) is singular.
    \item Substitute the given values of $a$ in to \mintinline{python}{A} and
        compute the inverse. Multiply the inverse by \(A\) to obtain the
        identity matrix $\begin{pmatrix}1 & 0 & \\ 0 & 1 & 0\\ 0 & 0 & 1\end{pmatrix}$
        which confirms the result.
\end{enumerate}

\section{Summary examples}
\hrule

\begin{multicols}{2}
    Create the matrix \(B = \begin{pmatrix} 3 & 5 \\ 1 & -2 \end{pmatrix}\).

        \begin{minted}[frame=single]{python}
import sympy as sym
B = sym.Matrix(((3, 5), (1, -2)))
        \end{minted}

    Obtain the determinant of \(B = \begin{pmatrix} 3 & 5 \\ 1 & -2 \end{pmatrix}\).


        \begin{minted}[frame=single]{python}
import sympy as sym
B = sym.Matrix(((3, 5), (1, -2)))
B.det()
        \end{minted}

    Obtain the inverse of \(B = \begin{pmatrix} 3 & 5 \\ 1 & -2 \end{pmatrix}\)

        \begin{minted}[frame=single]{python}
import sympy as sym
B = sym.Matrix(((3, 5), (1, -2)))
B.inv()
        \end{minted}

    Calculate \(\begin{pmatrix} 3 & 5 \\ 1 & -2
        \end{pmatrix}\left(\begin{pmatrix} 3 & 1 \\ 4 & 1 \end{pmatrix} +
            6\begin{pmatrix} 2 & 3 \\ 1 & 1 \end{pmatrix} \right)\)

        \begin{minted}[frame=single]{python}
import sympy as sym
B = sym.Matrix(((3, 5), (1, -2)))
C = sym.Matrix(((3, 1), (4, 1)))
D = sym.Matrix(((2, 3), (1, 1)))
B @ (C + 6 * D)
        \end{minted}

        Solve the linear system:
$$
    \begin{array}{l}
        x + 2y = 3\\
        3x + y= 4\\
    \end{array}
$$

        \begin{minted}[frame=single]{python}
import sympy as sym
M = sym.Matrix(((1, 2), (3, 1)))
b = sym.Matrix(((3,), (4,)))
M.inv() @ b
        \end{minted}
\end{multicols}

\end{document}
