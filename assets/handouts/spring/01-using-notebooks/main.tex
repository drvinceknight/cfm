\documentclass{article}

\usepackage[margin=1.5cm, includefoot, footskip=30pt]{geometry}
\usepackage[parfill]{parskip}
\usepackage[black]{merriweather} %% Option 'black' gives heavier bold face 
\usepackage[T1]{fontenc}
\usepackage{hyperref}
\usepackage{minted}

\pagenumbering{gobble}

\title{Computing for Mathematics: Handout 1}
\date{}

\begin{document}

\maketitle


This handout contains a summary of the topics covered and an activity to
carry out prior or during your lab session.

At the end of the handout is a specific coursework like exercise.

For further practice you can do the exercises available at 
\href{https://vknight.org/pfm/tools-for-mathematics/01-using-notebooks/exercises/main.html}{the Using
notebooks chapter of Python for Mathematics}.

\section{Summary}\label{summary}
\hrule


The purpose of this handout is to cover Using Notebooks which
corresponds to the
\href{https://vknight.org/pfm/tools-for-mathematics/01-using-notebooks/introduction/main.html}{Using
notebooks chapter of Python for Mathematics}.

The topics covered are:

\begin{itemize}
\item
  Installing Anaconda
\item
  Starting a notebook server
\item
  Creating a new notebook
\item
  Running some python code
\item
  Writing some markdown
\item
  Outputting notebooks to a different format
\end{itemize}


\section{Activity}\label{activity}
\hrule

There are instructions for how to do all of this is in the
\href{https://vknight.org/pfm/tools-for-mathematics/01-using-notebooks/how/}{How
to section of the Using Notebook chapter of Python for Mathematics}.

\begin{enumerate}
\item
  Install the Anaconda distribution.
\item
  Start a Jupyter notebook server.
\item
  Create a new notebook file.
\item
  Write some python code to compute \(2 + 2\).
\item
  Write the following in a markdown cell:


\begin{minted}[frame=single]{md}
As well as using Python in Jupyter notebooks we can also write using Markdown.
This allows us to use basic $\LaTeX$ as a way to display mathematics.
For example:

1. $\frac{2}{3}$
2. $\sum_{i=0}^n i$
\end{minted}

\item Save the notebook as an html file and open it.
\end{enumerate}


\section{Coursework like exercise}
\hrule

To ensure you are able to download and open the individual coursework when it
is released: download and open the notebook available at
\href{https://zenodo.org/record/7118738/files/demo.ipynb?download=1}{10.5281/zenodo.7118738}.

\end{document}
