\documentclass{article}

\usepackage[margin=1.5cm, includefoot, footskip=30pt]{geometry}
\usepackage[parfill]{parskip}
\usepackage[black]{merriweather} %% Option 'black' gives heavier bold face 
\usepackage[T1]{fontenc}
\usepackage{hyperref}
\usepackage{minted}
\usepackage{multicol}

\pagenumbering{gobble}

\title{Computing for Mathematics: Handout 2}
\date{}

\begin{document}

\maketitle


This handout contains a summary of the topics covered and an activity to
carry out prior or during your lab session.

At the end of the handout is a specific coursework like exercise.

For further practice you can do the exercises available at 
\href{https://vknight.org/pfm/tools-for-mathematics/02-algebra/exercises/main.html}{the
algebra chapter of Python for Mathematics}.

\section{Summary}\label{summary}
\hrule


The purpose of this handout is to cover Algebra which
corresponds to the
\href{https://vknight.org/pfm/tools-for-mathematics/02-algebra/introduction/main.html}{Algebra
chapter of Python for Mathematics}.

The topics covered are:

\begin{itemize}
\item
  Creating symbolic numeric values
\item
  Getting numerical value of a symbolic expression
\item
  Factorising an expression
\item
  Expanding an expression
\item
  Simplifying an expression
\item
  Solving an equation
\item
  Substituting values in to expressions
\end{itemize}


\section{Activity}\label{activity}
\hrule

We will be tackling the problem from the
\href{https://vknight.org/pfm/tools-for-mathematics/02-algebra/tutorial/}{tutorial
of the Algebra chapter of Python for Mathematics}.

\begin{enumerate}
    \item Rationalise the denominator of $\frac{1}{\sqrt{2} + 1}$
    \item Consider the quadratic: $f(x)=2x ^ 2 + x + 1$:
        \begin{enumerate}
            \item Calculate the discriminant of the quadratic equation $2x ^ 2 + x + 1 =
                 0$. What does this tell us about the solutions to the equation? What
                 does this tell us about the graph of $f(x)$?
         \item By completing the square, show that the minimum point of $f(x)$ is
     $\left(-\frac{1}{4}, \frac{7}{8}\right)$
        \end{enumerate}
\end{enumerate}

There are instructions for how to do all of this is in the
\href{https://vknight.org/pfm/tools-for-mathematics/02-algebra/how/}{Algebra chapter of Python for Mathematics}.


\begin{enumerate}
\item
  Create the variable \mintinline{python}{expression} which has value
        $\frac{1}{\sqrt{2}} + 1$.
\item Use the \mintinline{python}{sympy.simplify} command to rationalise the
        denominator.
\item Create the variable \mintinline{python}{expression} which has value the
        quadratic from the second part of the question: $f(x) = 2x^ 2 + x + 1$.
\item Use the \mintinline{python}{sympy.equation} and
        \mintinline{python}{sympy.sovleset} command to find the roots of $f$.
\item Create the variable \mintinline{python}{expression} which has value the
        expression $a(x-b) ^ 2 + c$.
\item Solve the various equations that give the correct values of $a, b$ and
    $c$ to be able to complete the square for $f(x)$.
\end{enumerate}


\section{Coursework like exercise}
\hrule


Consider the equation: $x ^ 2 + 4 - y = \frac{1}{y}$:

\begin{enumerate}
    \item Create a variable \mintinline{python}{general_solution} which has value the set of solutions to
   the equation for $x$ (as a function of $y$).
\item Create a variable \mintinline{python}{specific_solution} which has value the set of solutions when $y = 5$. 
\end{enumerate}

\section{Summary examples}
\hrule

\begin{multicols}{2}
        Create the symbolic value $1 / 3$

        \begin{minted}[frame=single]{python}
import sympy
value = 1 / sympy.S(3)
        \end{minted}

        Get the numeric value of a symbolic variable $1 / 3$

        \begin{minted}[frame=single]{python}
import sympy
float(value)
        \end{minted}

        Factor $x ^ 2 - 81$

        \begin{minted}[frame=single]{python}
import sympy
x = sympy.Symbol("x")
sympy.factor(x ** 2 - 81)
        \end{minted}

        Expand $(x - 1) (x + 1)$

        \begin{minted}[frame=single]{python}
import sympy
x = sympy.Symbol("x")
sympy.expand((x - 1) * (x + 1))
        \end{minted}

        Simplify $(x - 3) (x - 3)$

        \begin{minted}[frame=single]{python}
import sympy
x = sympy.Symbol("x")
sympy.simplify((x - 3) * (x - 3))
        \end{minted}

        Solve $x + 4 = x ^ 2$

        \begin{minted}[frame=single]{python}
import sympy
x = sympy.Symbol("x")
equation = sym.Eq(x + 4, x ** 2)
sympy.solveset(equation, x)
        \end{minted}

        Substitute $x=-2$ in to $x ^ 2 - 4$
        
        \begin{minted}[frame=single]{python}
import sympy
x = sympy.Symbol("x")
expression = x ** 2 - 4
expression.subs({x: -2})
        \end{minted}
\end{multicols}

\end{document}
