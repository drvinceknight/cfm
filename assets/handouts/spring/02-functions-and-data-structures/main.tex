\documentclass{article}

\usepackage[margin=1.5cm, includefoot, footskip=30pt]{geometry}
\usepackage[parfill]{parskip}
\usepackage[black]{merriweather} %% Option 'black' gives heavier bold face 
\usepackage[T1]{fontenc}
\usepackage{hyperref}
\usepackage{minted}

\pagenumbering{gobble}

\title{Computing for Mathematics: Handout 2}
\date{}

\begin{document}

\maketitle


This handout contains a summary of the topics covered as well as outline of
expected progress.

For further practice you can do the exercises available at 
\href{https://vknight.org/pfm/building-tools/02-functions-and-data-structures/exercises/main.html}{the
functions and data structures chapter of Python for Mathematics}.

\section{Expected progress}
\hrule

At the end of this week you should have a few potential ideas for projects.

\begin{itemize}
    \item For each of your ideas: what tools will be in your library?
    \item Have some early code for each of your ideas: this might help identify
        the coding techniques you will need to learn.
    \item Have an early conversation with me (Vince) about each idea.
\end{itemize}

\begin{center}
\framebox{
\textbf{At this stage I'd expect you to have some ideas and code written in a
    Jupyter notebook}
}
\end{center}

\section{Summary}\label{summary}
\hrule

The programming topics covered in the
\href{https://vknight.org/pfm/building-tools/02-functions-and-data-structures/introduction/main.html}{functions
and data structures} are:

% TODO

\begin{itemize}
\item
  Write docstrings:

        \begin{minted}[frame=single]{python}
def square(x):
    """
    Returns x ^ 2 + 1

    Parameters
    ----------
    x : float
        The element x

    Returns
    -------
    float
        The image
    """
    return x ** 2 + 1
        \end{minted}


\item
  Create a set:

        \begin{minted}[frame=single]{python}
unique_values = {"one", 2, "3"}
        \end{minted}


\item
  Do set operations:

        \begin{minted}[frame=single]{python}
unique_values = {"one", 2, "3"}
other_vaues = {1, 2, 3}
union = unique_values | other_values
intersection = unique_value & other_values
difference = unique_values - other_values
        \end{minted}

\item Using hash tables (called dictionaries in Python):

        \begin{minted}[frame=single]{python}
id_numbers = {"Vince": 839, "Julien": 20, "Kaitlynn": 1049}
id_numbers["Vince"]
        \end{minted}
\end{itemize}

\end{document}
