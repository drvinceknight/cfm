\documentclass{article}

\usepackage[margin=1.5cm, includefoot, footskip=30pt]{geometry}
\usepackage[parfill]{parskip}
\usepackage[black]{merriweather} %% Option 'black' gives heavier bold face 
\usepackage[T1]{fontenc}
\usepackage{hyperref}
\usepackage{minted}
\usepackage{multicol}

\pagenumbering{gobble}

\title{Computing for Mathematics: Handout 3}
\date{}

\begin{document}

\maketitle


This handout contains a summary of the topics covered and an activity to
carry out prior or during your lab session.

At the end of the handout is a specific coursework like exercise.

For further practice you can do the exercises available at 
\href{https://vknight.org/pfm/tools-for-mathematics/03-calculus/exercises/main.html}{the
calculus chapter of Python for Mathematics}.

\section{Summary}\label{summary}
\hrule


The purpose of this handout is to cover Calculus which
corresponds to the
\href{https://vknight.org/pfm/tools-for-mathematics/03-calculus/introduction/main.html}{Calculus
chapter of Python for Mathematics}.

The topics covered are:

\begin{itemize}
\item
  Getting the derivative of a symbolic expression.
\item
  Getting the indefinite integral of a symbolic expression.
\item
  Getting the definite integral of a symbolic expression.
\item
  Getting the limit of a symbolic expression.
\end{itemize}


\section{Activity}\label{activity}
\hrule

We will be tackling the problem from the
\href{https://vknight.org/pfm/tools-for-mathematics/03-calculus/tutorial/main.html}{tutorial
of the Calculus chapter of Python for Mathematics}.

Consider the function $f(x)= \frac{24 x \left(a - 4 x\right) + 2 \left(a - 8 x\right) \left(b - 4 x\right)}{\left(b - 4 x\right)^{4}}$

\begin{enumerate}
    \item Given that $\frac{df}{dx}|_{x=0}=0$, $\frac{d^2f}{dx^2}|_{x=0}=-1$ and
    that $b>0$ find the values of $a$ and $b$.
    \item For the specific values of $a$ and $b$ find:
        \begin{enumerate}
            \item $\lim_{x\to 0}f(x)$;
            \item $\lim_{x\to \infty}f(x)$;
            \item $\int f(x) dx$;
            \item $\int_{5}^{20} f(x) dx$.
        \end{enumerate}
\end{enumerate}

There are instructions for how to do all of this is in the
\href{https://vknight.org/pfm/tools-for-mathematics/03-calculus/how/main.html}{Calculus chapter of Python for Mathematics}.


\begin{enumerate}
\item
  Create the variable \mintinline{python}{expression} which has value
        $f(x)= \frac{24 x \left(a - 4 x\right) + 2 \left(a - 8 x\right) \left(b - 4 x\right)}{\left(b - 4 x\right)^{4}}$.
\item Use the \mintinline{python}{sympy.diff} command to obtain the
        derivative.
\item Create the variable \mintinline{python}{first_equation} which has value the
    equation that comes from the first condition of the question:
        $\frac{df}{dx}|_{x=0} = 0$.
\item Create the variable \mintinline{python}{second_equation} which has value the
    equation that comes from the second condition of the question:
        $\frac{d^2f}{dx^2}|_{x=0} = -1$.
    \item Solve both equations (use substitution if you helpful) and recalling
        that $b>0$ substitute the correct values of $a$ and $b$ in to
        \mintinline{python}{expression}.
    \item Obtain the required limits.
\end{enumerate}


\section{Coursework like exercise}
\hrule


Consider the second derivative $f''(x)=4 x + \cos(x)$.

\begin{enumerate}
    \item Create a variable \mintinline{python}{derivative} which has value 
        $f'(x)$ (use the variables \mintinline{python}{x} and
        \mintinline{python}{c1} if necessary):
    \item Create a variable \mintinline{python}{equation} that has value the 
        equation $f'(0)=0$.
    \item Using the solution to that equation, output the value of $\int_{0}^{5\pi}f(x)dx$.
\end{enumerate}

\section{Summary examples}
\hrule

\begin{multicols}{2}
    Calculate the second derivative of $\cos(x^2)$:

        \begin{minted}[frame=single]{python}
import sympy as sym
x = sym.Symbol("x")
expression = sym.cos(x ** 2)
sym.diff(expression, x, 2)
    \end{minted}

        Calculate the indefinite integral of $e^(x)$

        \begin{minted}[frame=single]{python}
import sympy as sym
x = sym.Symbol("x")
expression = sym.exp(x)
sym.integrate(expression, x)
        \end{minted}

        Calculate the definite integral $\int_0^{5}1/x$

        \begin{minted}[frame=single]{python}
import sympy as sym
x = sym.Symbol("x")
expression = 1 / x
sym.integrate(expression, (x, 0, 5))
        \end{minted}

        Obtain the limit $\lim_{h \to \infty}\frac{1}{cos^2(x)}$

        \begin{minted}[frame=single]{python}
import sympy
x = sympy.Symbol("x")
expression = 1 / (sym.cos(x) ** 2)
sym.limit(expression, x, sym.oo)
        \end{minted}
\end{multicols}

\end{document}
