\documentclass{article}

\usepackage[margin=1.5cm, includefoot, footskip=30pt]{geometry}
\usepackage[parfill]{parskip}
\usepackage[black]{merriweather} %% Option 'black' gives heavier bold face 
\usepackage[T1]{fontenc}
\usepackage{hyperref}
\usepackage{minted}

\pagenumbering{gobble}

\title{Computing for Mathematics: Handout 3}
\date{}

\begin{document}

\maketitle


This handout contains a summary of the topics covered as well as outline of
expected progress.

For further practice you can do the exercises available at 
\href{https://vknight.org/pfm/building-tools/03-object-oriented-programming/exercises/main.html}{the
object oriented programming chapter of Python for Mathematics}.

\section{Expected progress}
\hrule

At the end of this week you should have a clear project defined.

\begin{itemize}
    \item Have a concise answer to "What is the purpose of our library"?
    \item Have some general ideas about how your library would work.
    \item Have a list of similar and/or related tools.
\end{itemize}

\begin{center}
\framebox{
\textbf{At this stage I'd expect you to have a clear library in mind as well as
early code and documentation in Jupyter notebooks}
}
\end{center}

\section{Summary}\label{summary}
\hrule

The programming topics covered in the
\href{https://vknight.org/pfm/building-tools/03-object-oriented-programming/introduction/main.html}{functions
and data structures} are:

% TODO

\begin{itemize}
\item
  Writing a class:

        \begin{minted}[frame=single]{python}
class Polygon:
    """A class to represent a Polygon"""
    def __init__(self, number_of_sides):
        self.number_of_sides = number_of_sides

    def get_perimeter(self, length_of_side=None):
        """
        Computes the perimeter of the Polygon.

        If no side length is given, this assumes the unit length is the length
        of a given side.

        Parameters
        ----------
        length_of_side : int
            The length of a given side. The default is None.

        Returns
        -------
        float
            The perimeter of the polygon
        """
        return self.number_of_sides * length_of_side
        
        \end{minted}


\item
Create an instance of a class:

        \begin{minted}[frame=single]{python}
triangle = Polygon(number_of_sides=3)
        \end{minted}

\item
Call a method of a class:

        \begin{minted}[frame=single]{python}
triangle.get_perimeter(length_of_side=3)
        \end{minted}


\item
  Use inheritance

        \begin{minted}[frame=single]{python}
class Triangle(Polygon):
    """A class to represent polygons with 3 sides"""
    def __init__(self):
        self.number_of_sides = 3
        \end{minted}

\end{itemize}

\end{document}
