\documentclass{article}

\usepackage[margin=1.5cm, includefoot, footskip=30pt]{geometry}
\usepackage[parfill]{parskip}
\usepackage[black]{merriweather} %% Option 'black' gives heavier bold face 
\usepackage[T1]{fontenc}
\usepackage{hyperref}
\usepackage{minted}

\pagenumbering{gobble}

\title{Computing for Mathematics: Handout 5}
\date{}

\begin{document}

\maketitle


This handout contains a summary of the topics covered as well as outline of
expected progress.

For further practice you can do the exercises available at 
\href{https://vknight.org/pfm/building-tools/05-modularisation/exercises/main.html}{the
modularisation chapter of Python for Mathematics}.

\section{Expected progress}
\hrule

At the end of this week you should start modularising your code.

\begin{itemize}
    \item Ensure repeated code is isolated and modularised.
    \item Ensure code is in a \mintinline{python}{<library>.py} file.
\end{itemize}

\begin{center}
\framebox{
\textbf{At this stage I'd expect you to start creating your final \mintinline{python}{<library>.py}
    file.}
}
\end{center}

\section{Summary}\label{summary}
\hrule

The programming topics covered in the
\href{https://vknight.org/pfm/building-tools/05-modularisation/introduction/main.html}{modularisation
chapter
} are:

% TODO

\begin{itemize}
\item Importing code a python file.
  \item Breaking code up in to modular components.
\end{itemize}

\end{document}
