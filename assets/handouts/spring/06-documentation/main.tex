\documentclass{article}

\usepackage[margin=1.5cm, includefoot, footskip=30pt]{geometry}
\usepackage[parfill]{parskip}
\usepackage[black]{merriweather} %% Option 'black' gives heavier bold face 
\usepackage[T1]{fontenc}
\usepackage{hyperref}
\usepackage{minted}

\pagenumbering{gobble}

\title{Computing for Mathematics: Handout 6}
\date{}

\begin{document}

\maketitle


This handout contains a summary of the topics covered as well as outline of
expected progress.

For further practice you can do the exercises available at 
\href{https://vknight.org/pfm/building-tools/06-documentation/exercises/main.html}{the
documentation chapter of Python for Mathematics}.

\section{Expected progress}
\hrule

At the end of this week you should start writing your documentation.

\begin{itemize}
    \item Write your \mintinline{md}{README.md} file.
\end{itemize}

\begin{center}
\framebox{
\textbf{At this stage I'd expect you to have your final \mintinline{md}{README.md} file.}
}
\end{center}

\section{Summary}\label{summary}
\hrule

The programming topics covered in the
\href{https://vknight.org/pfm/building-tools/06-documentation/introduction/main.html}{documentation
chapter
} are:

% TODO

\begin{itemize}
\item Writing documentation using the Diataxis framework:
    \begin{itemize}
      \item Step by step example usage in a tutorial.
      \item Precise how to instructions in how to guides.
      \item Details of concepts in the explanation section.
      \item Background reading and information in the reference section.
    \end{itemize}
\end{itemize}

\end{document}
