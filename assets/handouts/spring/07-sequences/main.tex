\documentclass{article}

\usepackage{amsmath}
\usepackage[margin=1.5cm, includefoot, footskip=30pt]{geometry}
\usepackage[parfill]{parskip}
\usepackage[black]{merriweather} %% Option 'black' gives heavier bold face 
\usepackage[T1]{fontenc}
\usepackage{hyperref}
\usepackage{minted}
\usepackage{multicol}

\pagenumbering{gobble}

\title{Computing for Mathematics: Handout 7}
\date{}

\begin{document}

\maketitle


This handout contains a summary of the topics covered and an activity to
carry out prior or during your lab session.

At the end of the handout is a specific coursework like exercise.

For further practice you can do the exercises available at 
\href{https://vknight.org/pfm/tools-for-mathematics/07-sequences/introduction/main.html}{the
sequences chapter of Python for Mathematics}.

\section{Summary}\label{summary}
\hrule


The purpose of this handout is to cover sequences which
corresponds to the
\href{https://vknight.org/pfm/tools-for-mathematics/07-sequences/introduction/main.html}{probability
chapter of Python for Mathematics}.

The main topic covered here is recursion.

\section{Activity}\label{activity}
\hrule

We will be tackling the problem from the
\href{https://vknight.org/pfm/tools-for-mathematics/07-sequences/tutorial/main.html}{tutorial
of the sequences chapter of Python for Mathematics}.

A sequence $a_1, a_2, a_3, …$ is defined by:

$$
    \left\{
    \begin{array}{l}
        a_1 = k,\\
        a_{n + 1} = 2a_n – 7, n \geq 1,
    \end{array}
    \right.
$$

where $k$ is a constant.


\begin{enumerate}
    \item Write down an expression for $a_2$ in terms of $k$.
    \item Show that $a_3 = 4k -21$
    \item Given that $\sum_{r=1}^4 a_r = 43$ find the value of $k$.
\end{enumerate}

There are instructions for how to do all of this is in the
\href{https://vknight.org/pfm/tools-for-mathematics/07-sequences/how/main.html}{probability chapter of Python for Mathematics}.


\begin{enumerate}
    \item Define a python function \mintinline{python}{generate_a} which uses
        recursion to give the values of the sequence \(a_n\).
    \item Use a symbolic variable for \(k\) to obtain $a_1$, $a_2$, $a_3$ and
        $a_4$.
    \item Obtain the sum of these four values to get an equation for \(k\).
\end{enumerate}


\section{Coursework like exercise}
\hrule

Consider this recursive definition for the sequence $a_n$:

$$
a_n = \begin{cases}
        c & \text{ if n = 1}\\
        3a_{n - 1} + \frac{c}{n}
      \end{cases}
$$


\begin{enumerate}
    \item Output the sum of the 15 terms.
    \item Given that $c=2$ output $\frac{df}{dx}$ where:
             $$
             f(x) = a_1 + a_2 x + a_3 x ^ 2 + a_4 x ^ 3
             $$
    \item Given that $c=2$ output $\int f(x)dx$
\end{enumerate}


\section{Summary examples}
\hrule

    Define the following sequence:

    \[
        a_n = \begin{cases}
                1 & \text{ if }n=1\\
                \frac{1}{a_{n - 1} + 1}&\text{ otherwise}
              \end{cases}
    \]

        \begin{minted}[frame=single]{python}
def generate_a(n):
    """
    Generate the sequence a_n using recursion
    """
    if n == 1:
        return 1
    return 1 / (generate_a(n - 1) + 1)
    \end{minted}

\end{document}
