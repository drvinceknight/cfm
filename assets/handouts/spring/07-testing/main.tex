\documentclass{article}

\usepackage[margin=1.5cm, includefoot, footskip=30pt]{geometry}
\usepackage[parfill]{parskip}
\usepackage[black]{merriweather} %% Option 'black' gives heavier bold face 
\usepackage[T1]{fontenc}
\usepackage{hyperref}
\usepackage{minted}

\pagenumbering{gobble}

\title{Computing for Mathematics: Handout 7}
\date{}

\begin{document}

\maketitle


This handout contains a summary of the topics covered as well as outline of
expected progress.

For further practice you can do the exercises available at 
\href{https://vknight.org/pfm/building-tools/06-testing/exercises/main.html}{the
testing chapter of Python for Mathematics}.

\section{Expected progress}
\hrule

At the end of this week you should start writing your automated tests.

\begin{itemize}
    \item Write your \mintinline{python}{test_<library>.py} file.
    \item Include doctests in your \mintinline{md}{README.md} file.
\end{itemize}

\begin{center}
\framebox{
    \textbf{At this stage I'd expect you to have tests for your library.}
}
\end{center}

\section{Summary}\label{summary}
\hrule

The programming topics covered in the
\href{https://vknight.org/pfm/building-tools/06-testing/introduction/main.html}{testing
chapter
} are:

% TODO

\begin{itemize}
\item Write an assert statement.
\item Write and run a test file.
\item Write doctests.
\item Run doctests.
\end{itemize}

\end{document}
