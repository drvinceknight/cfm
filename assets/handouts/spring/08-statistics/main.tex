\documentclass{article}

\usepackage{amsmath}
\usepackage[margin=1.5cm, includefoot, footskip=30pt]{geometry}
\usepackage[parfill]{parskip}
\usepackage[black]{merriweather} %% Option 'black' gives heavier bold face 
\usepackage[T1]{fontenc}
\usepackage{hyperref}
\usepackage{minted}
\usepackage{multicol}

\pagenumbering{gobble}

\title{Computing for Mathematics: Handout 8}
\date{}

\begin{document}

\maketitle


This handout contains a summary of the topics covered and an activity to
carry out prior or during your lab session.

At the end of the handout is a specific coursework like exercise.

For further practice you can do the exercises available at 
\href{https://vknight.org/pfm/tools-for-mathematics/08-statistics/introduction/main.html}{the
statistics chapter of Python for Mathematics}.

\section{Summary}\label{summary}
\hrule


The purpose of this handout is to cover statistics which
corresponds to the
\href{https://vknight.org/pfm/tools-for-mathematics/08-statistics/introduction/main.html}{probability
chapter of Python for Mathematics}.

The topics covered are:

\begin{itemize}
    \item Calculating measures of central tendency and spread
    \item Calculating bivariate coefficients
    \item Fitting a line of best fit
    \item Using the Normal distribution
\end{itemize}
\section{Activity}\label{activity}
\hrule

We will be tackling the problem from the
\href{https://vknight.org/pfm/tools-for-mathematics/08-statistics/tutorial/main.html}{tutorial
of the statistics chapter of Python for Mathematics}.

Anna is investigating the relationship between exercise and resting heart rate.
She takes a random sample of 19 people in her year group and records for each person

- their resting heart rate, $h$ beats per minute.
- the number of minutes, $m$, spent exercising each week.

A table with the data is available in the statistics chapter of Python for
Mathematics where you can also see a scatter plot.

\begin{enumerate}
    \item For all collected values of $h$ and $m$ obtain:

    \begin{itemize}
        \item The mean
        \item The median
        \item The quartiles
        \item The standard deviation
        \item The variation
        \item The maximum
        \item The minimum
    \end{itemize}

\item Obtain the Pearson Coefficient of correlation for the variables $h$ and $m$.
\item Obtain the line of best fit for variables $x$ and $y$ as
   defined by:

   $$x=\ln(m)\qquad y=\ln(h)$$

\item Using the above obtain a relationship between $m$ and $h$ of the form:

   $$h=cm^k$$
\end{enumerate}

There are instructions for how to do all of this is in the
\href{https://vknight.org/pfm/tools-for-mathematics/08-statistics/how/main.html}{probability chapter of Python for Mathematics}.


\begin{enumerate}
    \item Create the variables \mintinline{python}{h} and \mintinline{python}{m}
        which have values the data for $h$ and $m$ respectively.
    \item Import the \mintinline{python}{statistics} library and use it to
        obtain the mean, media, quartiles, standard deviation and variation of
        both $h$ and $m$.
    \item Use the \mintinline{python}{min} and \mintinline{python}{max} tools to
        compute the minimum and maximum of both $h$ and $m$.
    \item Use the \mintinline{python}{statistics} library to compute the Pearson
        Coefficient of correlation between $h$ and $m$.
    \item Create \mintinline{python}{x} which has value $x=\ln(m)$.
    \item Create \mintinline{python}{y} which has value $y=\ln{h}$.
    \item Use the \mintinline{python}{statistics} library to compute the slope
        and intercept for a linear regression line between $y$ and $x$.
    \item Use \mintinline{python}{sympy} to obtain the required final expression
        for $h$ as a function of $m$.
\end{enumerate}


\section{Coursework like exercise}
\hrule


\begin{enumerate}
    \item Create a variable \mintinline{python}{distribution} that has value a
        Normal distribution with mean 10 and standard deviation .5:
    \item Create a variable \mintinline{python}{probability_of_less_than_eight}
        which has value the probability of selecting a random variable from the
        distribution which has value less than eight.
    \item Output the value of a variable from the distribution which has
        probability of being less than, .9.
\end{enumerate}

\section{Summary examples}
\hrule

\begin{multicols}{2}

    Calculate the sample standard deviation of $1, 4, 2, 3, 1.5, 7$:

        \begin{minted}[frame=single]{python}
import statistics as st
data = (1, 4, 2, 3, 1.5, 7)
st.stdev(data)
\end{minted}

Other tools exist in the statistics library to compute measures of spread and
tendency: \mintinline{python}{mean}, \mintinline{python}{median},
    \mintinline{python}{pstdev}, \mintinline{python}{stdev},
    \mintinline{python}{pvariance}, \mintinline{python}{variance},
    \mintinline{python}{quantiles}.

    Calculate the maximum and minimum of $1, 2, 3$:

        \begin{minted}[frame=single]{python}
data = (1, 2, 3)
max(data), min(data)
\end{minted}

    Obtain the covariance between $1, 2, 3$ and $3, 2, 1$:

        \begin{minted}[frame=single]{python}
import statistics as st
x = (1, 2, 3)
y = (3, 2, 1)
st.covariance(x, y)
\end{minted}

    Obtain the Pearson correlation coefficient between $1, 2, 3$ and $3, 2, 1$:

        \begin{minted}[frame=single]{python}
import statistics as st
x = (1, 2, 3)
y = (3, 2, 1)
st.correlation(x, y)
\end{minted}

    Fit a line of best fit between $1, 2.5, 3$ and $2.9, 2, 1$:

        \begin{minted}[frame=single]{python}
import statistics as st
x = (1, 2.5, 3)
y = (2.9, 2, 1)
st.linear_regression(x, y)
\end{minted}

    Create an instance of the normal distribution with mean $5$ and standard
    deviation $1$:

        \begin{minted}[frame=single]{python}
import statistics as st
st.NormalDist(mu=5, sigma=1)
\end{minted}

\end{multicols}

\end{document}
