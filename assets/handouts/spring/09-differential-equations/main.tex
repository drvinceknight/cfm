\documentclass{article}

\usepackage{amsmath}
\usepackage[margin=1.5cm, includefoot, footskip=30pt]{geometry}
\usepackage[parfill]{parskip}
\usepackage[black]{merriweather} %% Option 'black' gives heavier bold face 
\usepackage[T1]{fontenc}
\usepackage{hyperref}
\usepackage{minted}
\usepackage{multicol}

\pagenumbering{gobble}

\title{Computing for Mathematics: Handout 9}
\date{}

\begin{document}

\maketitle


This handout contains a summary of the topics covered and an activity to
carry out prior or during your lab session.

At the end of the handout is a specific coursework like exercise.

For further practice you can do the exercises available at 
\href{https://vknight.org/pfm/tools-for-mathematics/09-differential-equations/introduction/main.html}{the
differential equations chapter of Python for Mathematics}.

\section{Summary}\label{summary}
\hrule


The purpose of this handout is to cover differential equations which
corresponds to the
\href{https://vknight.org/pfm/tools-for-mathematics/09-differential-equations/introduction/main.html}{differential equations
chapter of Python for Mathematics}.

The topics covered are:

\begin{itemize}
    \item Creating a symbolic function
    \item Writing a differential equation
    \item Solving a differential equation
\end{itemize}
\section{Activity}\label{activity}
\hrule

We will be tackling the problem from the
\href{https://vknight.org/pfm/tools-for-mathematics/09-differential-equations/tutorial/main.html}{tutorial
of the differential equations chapter of Python for Mathematics}.

A container has volume $V$ of liquid which is poured in at a rate proportional
to $e^{-t}$ (where $t$ is some measurement of time). Initially the container is empty and
after $t=3$ time units the rate at which the liquid is poured is 15.

1. Show that $V(t)=\frac{-15e^{3}}{1-e^{3}}(1 - e^{-t})$
2. Obtain the limit $\lim_{t\to \infty}V(t)$

There are instructions for how to do all of this is in the
\href{https://vknight.org/pfm/tools-for-mathematics/09-differential-equations/how/main.html}{differential equations chapter of Python for Mathematics}.


\begin{enumerate}
    \item Create the symbolic variables \mintinline{python}{t} and
        \mintinline{python}{k} as well as the symbolic function
        \mintinline{python}{V}.
    \item Create the variable \mintinline{python}{differential_equation} which
    has value the differential equation $\frac{d}{dt}V(t)=ke^{-t}$    
    \item Use the \mintinline{python}{sympy.dsolve} tool to obtain the general
        solution to this differential equation.
    \item Use the initial conditions given (that $V(0)=0$) to obtain the
        particular solution to this differential equation.
    \item Use the fact that $V(3)=15$ to obtain a particular value for $k$.
    \item Obtain the required limit.
\end{enumerate}


\section{Coursework like exercise}
\hrule


\begin{enumerate}
    \item Create a variable \mintinline{python}{differential_equation} that has value a
        the differential equation: $\frac{dy}{dx}=\cos{y}$.
    \item Create a variable \mintinline{python}{general_solution} that has
        value the general solution (as an equation) for the differential equation.
    \item Create a variable \mintinline{python}{particular_solution} that has
        value the particular solution (as an equation) for the differential
        equation for the condition that $y(\pi)=5$.
\end{enumerate}

\section{Summary examples}
\hrule

\begin{multicols}{2}

    Create a symbolic function $g$:

        \begin{minted}[frame=single]{python}
import sympy as sym
g = sym.Function(g)
\end{minted}

    Create the differential equation $\frac{dy}{dx}=x$:

        \begin{minted}[frame=single]{python}
import sympy as sym
y = sym.Function(g)
x = sym.Function(x)

lhs = sym.diff(y(x), x)
differential_equation = sym.Eq(lhs, x)
\end{minted}

    \vspace{2cm}

    Solve the differential equation $\frac{dy}{dx}=x ^ 2$ given the condition
    $y(1)=0$

        \begin{minted}[frame=single]{python}
import sympy as sym
y = sym.Function(g)
x = sym.Function(x)

lhs = sym.diff(y(x), x)
rhs = x ** 2
differential_equation = sym.Eq(lhs, rhs)

condition = {y(1): 0}
solution = sym.dsolve(
    differential_equation,
    y(x),
    ics=condition,
)
\end{minted}

\end{multicols}

\end{document}
