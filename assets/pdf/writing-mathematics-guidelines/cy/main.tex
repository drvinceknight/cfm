\documentclass{article}

\usepackage{fullpage}
\usepackage{multicol}
\usepackage{amsmath}
\usepackage{amsfonts}
\usepackage{amsthm}
\usepackage{parskip}

\title{Rhai Canllawiau ar Gyfer Ysgrifennu Mathemategol Da}
\author{Francis Edward Su\\\small{Cyfieithu gan Geraint Palmer}}
\date{}

\renewcommand\familydefault{\sfdefault}

\begin{document}

\maketitle

\begin{multicols}{2}

Mae cyfathrebu mathemateg yn dda yn rhan bwysig o wneud mathemateg.
Fel mae nifer ohonom yn gwybod o ysgrifennu papurau neu'n rhoi cyflwyniadau, nid
yn unig yw cyfathrebu'n effeithiol yn helpu'r gynulleidfa, ond mae hefyd yn
egluro a strwythuro meddwl ein hunain.
Mae yna ceinder a chelf i ysgrifennu da y dylai pob awdur anelu ato.

Fan hyn, rydw i'n rhannu'r cyngor rydw i'n rhoi i'm myfyrwyr i i'w helpu nhw
ysgrifennu.
Mae adnoddau mwy eang (e.e. gweler ``How to Write Mathematics'' gan Paul
Halmos), ond hoffwn roi cyflwyniad byr.
Felly datblygais y canllawiau isod.

\section*{Sylfeini}

\textbf{Adnabod eich cynulleidfa.}
Hwn yw'r ystyriaeth fwyaf pwysig ar gyfer awduron.
Rhowch eich hunain o fewn esgidiau eich darllenwyr.
Pa gefndir gallwn dybio bod ganddynt?
Pa derminoleg sydd angen i ni ddiffinio?
Pa fath o ``lais'' ydych chi eisiau taflu: ffurfiol neu'n anffurfiol, difrifol
neu'n ddeniadol, cryno neu'n siaradus?

Os ydych chi'n fyfyriwr yn ysgrifennu datrysiadau ar gyfer gwaith cartref a
rhoddir gan athro sydd heb nodi pwy yw'r gynulleidfa, rheol dda yw tybio eich
bod yn ysgrifennu ar gyfer myfyriwr arall ar y cwrs sydd heb wneud yr aseiniad.
Er gallwch dybio eu bod wedi mynychu'r un darlithoedd a darllen yr un gwerslyfr
a chi, mae'n fater o gwrteisi i atgoffa'ch darllenwyr o unrhyw eitemau
perthnasol rydych wedi dysgu'n ddiweddar yn y dosbarth neu yn y gwerslyfr, neu
bethau efallai maent wedi anghofio.

Er enghraifft, os oedd cysyniad rhif cymarebol wedi'i dysgu'n ddiweddar, efallai
byddwch yn ychwanegu ``Cofiwch gall ysgrifennu rhif cymarebol fel ffracsiwn'',
cyn dweud ``gan fod $x$ yn gymarebol, $x = m/n$ lle mae $m$ ac $n$ yn
gyfanrifau.''

\vspace{4mm}

\textbf{Gosod t\^{o}n gwahoddiol.}
Mae'r traddodiadol i greu awyrgylch gwahoddiol o fewn ysgrifennu mathemategol.
Mewn effaith, rydym yn gwahodd darllenwyr i ymuno a'n broses resymeg trwy
ysgrifennu yn yr amser presennol, trwy ddefnyddio'r rhagenw ``ni'' yn lle ``fi''
(e.e. ``rydym yn adeiladau pl\^{a}n tangiadol...''), a trwy gyfarwyddo'r
darllenwyr gyda gorchmynion ysgafn (e.e. ``gadewch i $n$...'', ``cofiwch
fod...'', neu ``ystyriwch y set o...'').

\vspace{4mm}

\textbf{Defnyddiwch frawddegau llawn.}
Dyle pob darn o fathemateg cael eu hysgrifennu mewn brawddegau.
Agorwch unrhyw werslyfr mathemateg a gwelwch fod hwn yn wir.
Mae gan hafaliadau, hyd yn oed rhai arddangosiedig, atalnodi sy'n helpu ni gweld
ble maent yn ffitio mewn i gyd-destun brawddegau mwy.
Ystyriwch y darn o ysgrifennu canlynol:

\begin{align*}
(x - 2)^2 + (x - 1)^2 &= 5^2 5^2 = 25\\
(x - 2)^2 = x^2 - 4x + 4 &+ x^2 - 2x + 1 = 25.\\
2x^2 - 6x &- 20\\
2(x + 2)(x - 5)x = &-2, 5 \quad x > 0 \quad x = 5
\end{align*}

Allwch chi weithio allan beth mae'r awdur yn gwneud?
Beth sydd wedi cael eu tybio?
Beth sy'n cael eu profi?
Ble mae un syniad yn gorffen ac un arall yn dechrau?
Beth yw'r perthynas rhwng yr ymadroddion?
Mae rhai ymadroddion yn anghyflawn, ac nid yw rhai yn wir hyd yn oed.
Ni ddylai'r darllenwr gorfod gweithio allan beth oedd yr awdur yn meddwl.

Nawr ystyriwch waith awdur arall yn ceisio'r un broblem:

\setlength{\leftskip}{0.7cm}

\textbf{Problem.} Canfyddwch bwynt ar y llinell $y = x$ sy'n bellter 5 o'r pwynt
(2, 1) ac mae ei chyfesuryn-$x$ yn bositif.

\textbf{Datrysiad.} Y pwynt dymunol yw (5, 5).
I weld hwn, rydym yn datrys $(x - 2)^2 + (x - 1)^2 = 5^2$, hafaliad a gafwyd o'r
fformiwla pellter yn y pl\^{a}n.
Mae bach o algebra yn troi'r hafaliad yma i mewn i:

\begin{equation*}
2x^2 - 6x - 20 = 0.
\end{equation*}

Yn ffactoreiddio'r ochr chwith, cawn

\begin{equation*}
2(x + 2)(x - 5) = 0,
\end{equation*}

a'i datrysiadau yw $x = -2$ ac $x = 5$.
Oherwydd tybion ni fod $x > 0$, cawn (5, 5) fel y pwynt dymunol ar y llinell
$y = x$.

\setlength{\leftskip}{0pt}

Fan hyn mae'r awdur wedi nodi'r broblem yn glir ac wedi disgrifio'i llwybr i'w
datrysiad.
Mae wedi gosod t\^{o}n gwahoddiol, ac mae pob syniad wedi'i mynegi mewn
brawddegau llawn.
Mae'n glir bod $x > 0$ yn amod, nid yn ganlyniad.
Sylwch ar ei atalnodi yn ei hafaliadau: fe wnaeth un bennu gydag atalnod llawn
gan fod ei syniad yn gyflawn, ac fe wnaeth un arall bennu gydag atalnod oherwydd
roedd hi eisiau parhau gyda'r syniad.

Oherwydd tybiodd hi fod ei chynulleidfa yn gallu gwneud algebra, ni ddiflasodd
hi nhw gyda'r driniaeth algebra, a fydd yn cuddio llif y ddadleuon.
Ond fe wnaeth hi dangos y darn hanfodol a'r darn fwyaf diddorol: y ffactoreiddio
a'r canlyniad.
A sicrhaodd ei bod wedi ateb y cwestiwn gwreiddiol.

\vspace{4mm}

\textbf{Defnyddiwch eiriau i roi cyd-destun i hafaliadau.}
Ystyriwch y gwahaniaeth ystyr rhwng y tri datganiad: ``Gadawer i $A = 5$'',
``Tybiwch fod $A = 5$'', ac ``Felly mae $A = 5$''.
Yna meddyliwch am ba mor amwys yw'r datganiad ``$A = 5$''.

\vspace{4mm}

\textbf{Osgoi llaw-fer mewn ysgrifennu ffurfiol.}
Mae nifer fawr o fathau o ysgrifennu mathemategol, a gallwn ei grwpio yn fras i
mewn i ysgrifennu ffurfiol ac anffurfiol.
Mae ysgrifennu anffurfiol yn cynnwys ysgrifennu ar fwrdd du yn ystod darlith,
neu esbonio rhywbeth i ffrind ar ddarn o bapur sgrap.
Mae ysgrifennu ffurfiol yn cynnwys y math o ysgrifennu a ddisgwylir am
aseiniadau gwaith cartref, neu mewn papur academaidd.
Mae gwahaniaethau o ran beth sy'n dderbyniol.
Er enghraifft mewn ysgrifennu anffurfiol mae'n gyffredin iawn i ddefnyddio
llaw-fer ar gyfer meintiolwyr a goblygiadau: symbolau megis $\forall$,
$\exists$, $\Rightarrow$, $\Leftrightarrow$, neu fyrfoddau megis ``iff'' a
``s.t.''.
Ond, mewn ysgrifennu ffurfiol, dylen osgoi llaw-fer fel hyn yn gyffredinol.
Dylen ysgrifennu allan ``ar gyfer pob'', ``bodoler'', ``yn awgrymu'', ``os ac yn
unig os'', ac ``fel bod''.

Mae rhan fwyaf o symbolau eraill yn dderbyniol mewn ysgrifennu ffurfiol, ar
\^{o}l eu diffinio lle mae angen.
Mae'r symbol aelodaeth $\in$ yn draddodiadol yn dderbyniol mewn ysgrifennu
ffurfiol, yn ogystal \^{a} pherthnasau (e.e. $<$, $+$, $\cup$, ayyb.), enwau
newidynnau (e.e. $x$, $y$, $z$), a symbolau ar gyfer setiau (e.e. $\mathbb{R}$).
Dyma ddefnydd derbyniol symbolau mewn ysgrifennu mathemategol ffurfiol:

\setlength{\leftskip}{0.7cm}

Gadawer i $A$ a $B$ bod yn is-setiau o $\mathbb{R}$.
Dywedwn fod $A$ \textit{yn dominyddu} $B$ os ar gyfer pob $x \in A$ bodoler
$y \in B$ fel bod $y > x$.

\setlength{\leftskip}{0pt}

\vspace{4mm}

\textbf{Dysgwch yr arferion.}
Mae'r enghraifft uchod hefyd yn dangos dau gonfensiwn cyffredin arferion
mathemategol.
Mae'r arferol peidio dechrau brawddegau neu ymadroddion gyda rhif neu symbol
oherwydd gall hyn fod yn ddryslyd.
Mae hefyd yn arferol i bwysleisio geiriau anghyfarwydd y rydym ar fin diffinio,
er enghraifft trwy ei rhoi yn italig.
Gall dysgu arferion eraill trwy arsylwi ar y normau a ddefnyddiwyd yn eich maes
astudiaeth.


\section*{Tuag at ceinder}

\textbf{Penderfynwch beth sy'n bwysig i'w ddweud.}
Nid yw ysgrifennu'n dda yn angenrheidiol yn golygu ysgrifennu mwy.
Os yw eich datrysiad yn rhy eiriol, gall hwn weithiau cuddio'r pwyntiau rydych
yn ceisio i'w gwneud.

Mae datrysiad sydd wedi'i ysgrifennu'n dda yn cyflwyno jyst digon o fanylion a'r
darnau mwyaf diddorol y ddadl.
Pa theoremau neu acsiomau oedd yn hanfodol wrth gael eich datrysiad, a phryd a
ble y defnyddiwyd?
Nid eich prif r\^{o}l chi fel awdur yw rhoi manylion (er bod hynny'n bwysig
iawn).
Eich prif r\^{o}l chi yw rhoi mewnwelediad.

\vspace{4mm}

\textbf{Uwcholeuwch strwythur.}
Os yw eich dadl yn un hir, gyda nifer mawr o fanylion technegol, yna helpwch
eich darllenwyr trwy grynhoi amlinelliad eich dadl ar y dechrau.
Yna, trwy gydol yr ysgrifennu, helpwch eich darllenwyr gweld sut ydych yn symud
ymlaen trwy'r amlinelliad hynny.

\vspace{4mm}

\textbf{Defnyddiwch baragraffau i bwysleisio blociau o syniadau sy'n perthyn.}
Mae swydd y frawddeg gyntaf mewn paragraff yn hollbwysig: dychmygwch ddarllenwr
yn sgimio dros eich ysgrifennu, ond yn darllen y brawddegau cyntaf.
A fyddai'n gallu dilyn llif y ddadl?
Yn debyg, efallai byddwch ond eisiau dangos yr hafaliadau mwyaf pwysig.
Mae amnewid rhyw ddadl sy'n cael ei ailadrodd tro ac \^{o}l tro gyda lema da yn
gallu symleiddio'r llif yn ogystal \^{a} uwcholeuo syniad allweddol.

\vspace{4mm}

\textbf{Dewiswch enghreifftiau da.}
Fe ellir gwneud rhyw syniad anodd yn haws i'w ddeall os rhoddir enghraifft ochr
yn ochr efo.
Dewiswch un enghraifft digon syml i'w ddilyn, ond sy'n ddigon diddorol i gadw'r
nodweddion pwysig.
Gall prawf o rhyw syniad cyffredinol dod cyn rhyw enghraifft sy'n ei defnydddio
mewn cyd-destun penodol.
Gallwn wella rhyw amlygiad hir trwy ddefnyddio enghraifft barhaol - un lle
defnyddir yr un enghraifft mewn nifer o gyd-destunau.

\vspace{4mm}

\textbf{Osgoi ysgadan sych!}
\textit{Gadewch allan unrhyw fanylion nad oes ganddynt unrhyw bwys i ddatrysiad
y broblem, oherwydd gallant gamarwain y darllenydd.}
Er enghraifft, os dywedwn ``gallwn fynegi'r rhif cymarebol $r$ fel $m/n$ lle nad
oes gan $m$ ac $n$ unrhyw ffactorau cyffredin'', rydych yn gadael cliw y byddwch
yn defnyddio'r syniad nad oes ``unrhyw ffactorau cyffredin'' nes ymlaen.
Felly os nad ydych yn defnyddio'r ffaith yma, ni ddylech ei ddweud.
Mae'n amherthnasol.
Efallai bod ysgadan sych gwneud nofelau dirgelwch yn hwyl, ond mewn ysgrifennu
mathemategol eich g\^{o}l yw cael gwared a'r dirgelwch!


\vspace{4mm}

\textbf{Cymerwch gam yn \^{o}l a symleiddiwch.}
Ar \^{o}l ysgrifennu prawf, cymerwch gam yn \^{o}l a gofynnwch:
Sut allaf i symleiddio'r ddadl yma?
A ddefnyddiais i bob offeryn yma i ddatrys y broblem?
A allaf i symleiddio'r ddadl?
Er enghraifft, ystyriwch y prawf trwy wrthddywediad canlynol:

\setlength{\leftskip}{0.7cm}

\textbf{Problem.}
Dangoswch os yw 4 yn rhannu cyfanrif $n$, yna mae $n$ yn eilrif.

\textit{Prawf.}
Tybiwch nad yw $n$ yn eilrif.
Gan fod 4 yn rhannu ag $n$, mae gennym $n = 4k$ ar gyfer rhyw gyfanrif $k$.
Felly $n = 2(2k)$, sy'n eilrif.
Mae hwn yn gwrthddweud ein damcaniaeth gwreiddiol nad yw $n$ yn eilrif. QED.

\setlength{\leftskip}{0pt}

A welwch pam nad yw hwn wir yn brawf trwy wrthddywediad?
Ni ddefnyddiwyd y damcaniaeth anghyson gwreiddiol!
Tynnwch i ffwrdd y frawddeg gyntaf a'r frawddeg olaf, ac mae gennym brawf
uniongyrchol, cain.

\vspace{4mm}

\textbf{Mireiniwch, mireiniwch, mireiniwch.}
Mae ysgrifennu da yn broses o frasamcanion olynol.
Ni ddylech chi ddisgwyl i'ch drafft cyntaf fod yn berffaith.
Wrth i chi adolygu'ch ysgrifennu byddwch yn gweld ffyrdd o grynhoi dadl neu i'w
ddweud mewn ffordd well.
Dyma'r rhan o'r broses ysgrifennu a fydd yn helpu chi egluro'ch meddwl eich
hunain hefyd.

Fel arfer, ar \^{o}l cwblhau drafft, bydd awdur yn sylwi ar ryw ddewis mewn
nodiant neu ddiffiniad nad yw'n gorau bosibl.
Bydd awdur diog yn gadael pethau fel y mae, ond bydd awdur meddylgar yn cymryd
amser i fynd n\^{o}l a gwneud newidiadau.

\vspace{4mm}

\textbf{Arsylwch ar y diwylliant.}
Nid yw cyfathrebu da yn arwahanol i'r diwylliant lle mae'n digwydd.
Efallai bod y sylweddiad yma yn anghyfforddus i fathemategwyr buddiol, sydd wedi
denu i natur resymegol ac absoliwt mathemateg.
Ond mae hyd yn oed yr absoliwtau yma wedi'i mynegi'n wahanol mewn oesoedd
gwahanol, cymharwch ysgrifennu Newton i werslyfrau calcwlws heddiw.
Mae rheolau arferion mathemateg wedi esblygu.

Er bod y canllawiau yma yn ceisio uwcholeuo egwyddorion ysgrifennu ffurfiol, mae
trwy'r amser eithriadau - gan fod gan rhai meysydd mathemategol normau
gwahanol i'w gilydd.
Y ffordd gorau i gael teimlad o beth sy'n dderbyniol yn eich cyd-destun chi yw
pori dros nifer o destunau uchel eu parch neu bapurau sy'n gysylltiedig \^{a}'r
ddogfen rydych yn ysgrifennu.

\vspace{4mm}

\textbf{Mwynhewch ysgrifennu.}
Mae ysgrifennu yn amser i adlewyrchu ar syniadau prydferth a'i beintio ar canfas
gyda gofal artistig manwl.

\vspace{4mm}

\textbf{Cydnabyddiadau.}
Mae'n bwysig adnabyddi'r cymorth rydych wedi derbyn.
Mae fy adolygiadau o canllawiau yma wedi buddio o'r adborth defnyddiol gan Jon
Jacobsen a Lesley Ward.

\end{multicols}
\end{document}