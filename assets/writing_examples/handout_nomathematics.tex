\documentclass[12pt]{article}

% \usepackage{fullpage}
\usepackage{geometry}
\geometry{margin=0.8in, top=1in, bottom=0.9in}
\usepackage{parskip}
\usepackage{setspace}
\usepackage{mathtools}
\usepackage{enumerate}
\usepackage{multicol}
\usepackage{booktabs}
\usepackage[T1,hyphens]{url}
\usepackage{hyperref}
\usepackage{minitoc}
\usepackage{standalone}
\usepackage{tikz}
\usetikzlibrary{arrows}
\usetikzlibrary{decorations.markings}
\usetikzlibrary{calc}
\usetikzlibrary{shapes,snakes}
\usepackage{array}
\usepackage{longtable}
\usepackage{pdfpages}
\usepackage{array}
\usepackage{pgfplots}
\usepackage{pgfplotstable}
\usepackage{color, colortbl}
\usepackage{diagbox}
\usepackage{titlesec}
\usepackage[gen]{eurosym}
\usepackage{multirow}
\usepackage{tcolorbox}
\usepackage{amsfonts}
\usepackage[shortlabels]{enumitem}
\usepackage{minted}
\usepackage[ruled]{algorithm2e}
\usepackage{xfrac}
\usepackage[shortlabels]{enumitem}
\usepackage{titlesec, blindtext, color}
\usepackage{fancyhdr}
\usepackage{mdframed}

\setlength{\headsep}{10mm}

\definecolor{gray75}{gray}{0.75}
\newcommand{\hsp}{\hspace{20pt}}

\titleformat{\chapter}[hang]{\LARGE\bfseries}{\thechapter\hsp\textcolor{gray75}{$\vline$}\hsp}{0pt}{\LARGE\bfseries}

\renewcommand\familydefault{\sfdefault}

\newcommand{\specialcelll}[2][l]{%
  \begin{tabular}[#1]{@{}l@{}}#2\end{tabular}}


\definecolor{codebg}{RGB}{255, 255, 230}
\setminted[python]{
    frame=single,
    framesep=2mm,
    bgcolor=codebg,
    framerule=0.5mm,
    fontsize=\small,
}

\BeforeBeginEnvironment{minted}{\vspace{-3mm}}
\AfterEndEnvironment{minted}{\vspace{-3mm}}


\nomtcrule

\begin{document}

\begin{center}
\LARGE{\textit{Tin Can Example - Writing Only}}
\end{center}

\vspace{15mm}

\textbf{\textit{QUESTION:}}

\textbf{Consider a cylindrical tin can with radius $r$ and height $h$. Let its volume $V$ be non-zero and fixed. Find the relationship between $h$ and $r$ such that the surface area of the tin can is minimised.}

\vspace{5mm}
\hrule
\vspace{5mm}

\textbf{\textit{SOLUTION:}}

\vspace{5mm}

\begin{mdframed}[linewidth=0.5mm, backgroundcolor=orange!10]
First, let $r$ be the radius of the cylinder, $h$ be its height, and $V$ be its volume. We have expressions for the tin's volume and surface area:
\end{mdframed}

\vspace{10.5mm}

\begin{mdframed}[linewidth=0.5mm, backgroundcolor=orange!10]
As the volume $V$ is fixed, then we have a relationship between $r$ and $h$ that is dependant on $V$:
\end{mdframed}

\vspace{10.5mm}

\begin{mdframed}[linewidth=0.5mm, backgroundcolor=orange!10]
Therefore the surface area now becomes a function of $V$ and $r$:
\end{mdframed}

\vspace{10.5mm}

\begin{mdframed}[linewidth=0.5mm, backgroundcolor=orange!10]
The surface area is minimised when its derivative is equal to zero:
\end{mdframed}

\vspace{10.5mm}

\begin{mdframed}[linewidth=0.5mm, backgroundcolor=orange!10]
Then solving gives $\tilde{r}$, the value of $r$ that minimises the surface area, in terms of $V$:
\end{mdframed}

\vspace{10.5mm}

\begin{mdframed}[linewidth=0.5mm, backgroundcolor=orange!10]
However, as mentioned above, the volume $V$ is fixed, and is itself a function of $r$ and $h$. Substituting this in gives an implicit relationship between $\tilde{r}$ and $h$:
\end{mdframed}

\vspace{10.5mm}

\begin{mdframed}[linewidth=0.5mm, backgroundcolor=orange!10]
Solving for either $\tilde{r}$ or $h$ will give an explicit relationship between $\tilde{r}$ and $h$, as required.
\end{mdframed}

\vspace{10.5mm}

\begin{mdframed}[linewidth=0.5mm, backgroundcolor=orange!10]
Therefore, for a fixed non-zero volume cylinder, the relationship between the radius $r$ and height $h$ that minimised the surface area is $h = 2r$.
\end{mdframed}



\end{document}