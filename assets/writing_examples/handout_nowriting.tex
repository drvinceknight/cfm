\documentclass[12pt]{article}

% \usepackage{fullpage}
\usepackage{geometry}
\geometry{margin=0.8in, top=1in, bottom=0.9in}
\usepackage{parskip}
\usepackage{setspace}
\usepackage{mathtools}
\usepackage{enumerate}
\usepackage{multicol}
\usepackage{booktabs}
\usepackage[T1,hyphens]{url}
\usepackage{hyperref}
\usepackage{minitoc}
\usepackage{standalone}
\usepackage{tikz}
\usetikzlibrary{arrows}
\usetikzlibrary{decorations.markings}
\usetikzlibrary{calc}
\usetikzlibrary{shapes,snakes}
\usepackage{array}
\usepackage{longtable}
\usepackage{pdfpages}
\usepackage{array}
\usepackage{pgfplots}
\usepackage{pgfplotstable}
\usepackage{color, colortbl}
\usepackage{diagbox}
\usepackage{titlesec}
\usepackage[gen]{eurosym}
\usepackage{multirow}
\usepackage{tcolorbox}
\usepackage{amsfonts}
\usepackage[shortlabels]{enumitem}
\usepackage{minted}
\usepackage[ruled]{algorithm2e}
\usepackage{xfrac}
\usepackage[shortlabels]{enumitem}
\usepackage{titlesec, blindtext, color}
\usepackage{fancyhdr}
\usepackage{mdframed}

\setlength{\headsep}{10mm}

\definecolor{gray75}{gray}{0.75}
\newcommand{\hsp}{\hspace{20pt}}

\titleformat{\chapter}[hang]{\LARGE\bfseries}{\thechapter\hsp\textcolor{gray75}{$\vline$}\hsp}{0pt}{\LARGE\bfseries}

\renewcommand\familydefault{\sfdefault}

\newcommand{\specialcelll}[2][l]{%
  \begin{tabular}[#1]{@{}l@{}}#2\end{tabular}}


\definecolor{codebg}{RGB}{255, 255, 230}
\setminted[python]{
    frame=single,
    framesep=2mm,
    bgcolor=codebg,
    framerule=0.5mm,
    fontsize=\small,
}

\BeforeBeginEnvironment{minted}{\vspace{-3mm}}
\AfterEndEnvironment{minted}{\vspace{-3mm}}


\nomtcrule

\begin{document}

\begin{center}
\LARGE{\textit{Tin Can Example - Mathematical Notation Only}}
\end{center}

\vspace{15mm}

\begin{minipage}[t]{\textwidth}
\begin{mdframed}[linewidth=0.5mm, backgroundcolor=cyan!10]
\begin{align*}
V &= \pi r^2 h \\
S &= 2\pi r^2 + 2 \pi rh
\end{align*}
\end{mdframed}
\end{minipage}

\vspace{10mm}

\begin{minipage}[t]{\textwidth}
\begin{mdframed}[linewidth=0.5mm, backgroundcolor=cyan!10]
\begin{align*}
V &= \pi r^2 h \\
\frac{V}{\pi r^2} &= h \\
\end{align*}
\end{mdframed}
\end{minipage}

\vspace{10mm}

\begin{minipage}[t]{\textwidth}
\begin{mdframed}[linewidth=0.5mm, backgroundcolor=cyan!10]
\begin{align*}
S &= 2\pi r^2 + 2 \pi rh \\
&= 2\pi r^2 + 2 \pi r\left(\frac{V}{\pi r^2}\right) \\
&= 2\pi r^2 + \frac{2 V}{r} \\
\end{align*}
\end{mdframed}
\end{minipage}

\vspace{10mm}

\begin{minipage}[t]{\textwidth}
\begin{mdframed}[linewidth=0.5mm, backgroundcolor=cyan!10]
\begin{align*}
0 &= \frac{dS}{dr} \\
&= 4\pi r - 2Vr^{-2}
\end{align*}
\end{mdframed}
\end{minipage}

\vspace{10mm}

\begin{minipage}[t]{\textwidth}
\begin{mdframed}[linewidth=0.5mm, backgroundcolor=cyan!10]
\begin{align*}
4\pi \tilde{r} - \frac{2V}{\tilde{r}^{2}} &= 0 \\
4\pi \tilde{r} &= \frac{2V}{\tilde{r}^{2}} \\
4\pi \tilde{r}^3 &= 2V \\
\tilde{r}^3 &= \frac{V}{2\pi} \\
\tilde{r} &= \sqrt[3]{\frac{V}{2\pi}} \\
\end{align*}
\end{mdframed}
\end{minipage}

\vspace{10mm}

\begin{minipage}[t]{\textwidth}
\begin{mdframed}[linewidth=0.5mm, backgroundcolor=cyan!10]
\begin{align*}
\tilde{r} &= \sqrt[3]{\frac{V}{2\pi}} \\
&= \sqrt[3]{\frac{\pi \tilde{r}^2 h}{2\pi}} \\
&= \sqrt[3]{\frac{\tilde{r}^2 h}{2}}
\end{align*}
\end{mdframed}
\end{minipage}

\vspace{10mm}

\begin{minipage}[t]{\textwidth}
\begin{mdframed}[linewidth=0.5mm, backgroundcolor=cyan!10]
\begin{align*}
\tilde{r} &= \sqrt[3]{\frac{\tilde{r}^2 h}{2}} \\
\tilde{r}^3 &= \frac{\tilde{r}^2 h}{2} \\
2\tilde{r}^3 &= \tilde{r}^2 h \\
2\tilde{r} &= h \\
\end{align*}
\end{mdframed}
\end{minipage}


\end{document}