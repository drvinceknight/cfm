\documentclass[12pt]{article}

% \usepackage{fullpage}
\usepackage{geometry}
\geometry{margin=0.8in, top=1in, bottom=0.9in}
\usepackage{parskip}
\usepackage{setspace}
\usepackage{mathtools}
\usepackage{enumerate}
\usepackage{multicol}
\usepackage{booktabs}
\usepackage[T1,hyphens]{url}
\usepackage{hyperref}
\usepackage{minitoc}
\usepackage{standalone}
\usepackage{tikz}
\usetikzlibrary{arrows}
\usetikzlibrary{decorations.markings}
\usetikzlibrary{calc}
\usetikzlibrary{shapes,snakes}
\usepackage{array}
\usepackage{longtable}
\usepackage{pdfpages}
\usepackage{array}
\usepackage{pgfplots}
\usepackage{pgfplotstable}
\usepackage{color, colortbl}
\usepackage{diagbox}
\usepackage{titlesec}
\usepackage[gen]{eurosym}
\usepackage{multirow}
\usepackage{tcolorbox}
\usepackage{amsfonts}
\usepackage[shortlabels]{enumitem}
\usepackage{minted}
\usepackage[ruled]{algorithm2e}
\usepackage{xfrac}
\usepackage[shortlabels]{enumitem}
\usepackage{titlesec, blindtext, color}
\usepackage{fancyhdr}
\usepackage{mdframed}

\setlength{\headsep}{10mm}

\definecolor{gray75}{gray}{0.75}
\newcommand{\hsp}{\hspace{20pt}}

\titleformat{\chapter}[hang]{\LARGE\bfseries}{\thechapter\hsp\textcolor{gray75}{$\vline$}\hsp}{0pt}{\LARGE\bfseries}

\renewcommand\familydefault{\sfdefault}

\newcommand{\specialcelll}[2][l]{%
  \begin{tabular}[#1]{@{}l@{}}#2\end{tabular}}


\definecolor{codebg}{RGB}{255, 255, 230}
\setminted[python]{
    frame=single,
    framesep=2mm,
    bgcolor=codebg,
    framerule=0.5mm,
    fontsize=\small,
}

\BeforeBeginEnvironment{minted}{\vspace{-3mm}}
\AfterEndEnvironment{minted}{\vspace{-3mm}}


\nomtcrule

\begin{document}

\begin{center}
\LARGE{\textit{Writing Mathematics}}
\end{center}


\begin{center}\textit{We have not ``done mathematics'' unless we have communicated it.}\end{center}

\vspace{5mm}

\begin{mdframed}[linewidth=0.5mm, backgroundcolor=cyan!10]
Mathematical notation ($4+1$, $f(x) = x^2$, $\sin(x) \approx x$, etc.) is itself a language - but it cannot tell us the full story! It is useful for communicating:
\begin{itemize}
  \item \textbf{what} calculations were done,
  \item \textbf{what} manipulations were done,
  \item \textbf{how} manipulations were done,
  \item \textbf{what} relationships exist.
\end{itemize}
\end{mdframed}

\vspace{5mm}

\begin{mdframed}[linewidth=0.5mm, backgroundcolor=orange!10]
But we still need \textit{words} (in English / Welsh / Spanish / Chinese, etc.) to:
\begin{itemize}
  \item tell us \textbf{why} calculations were done,
  \item tell us \textbf{why} manipulations were done,
  \item \textbf{motivate} why we care that certain relationships exist,
  \item \textbf{interpret} results and relationships,
  \item \textbf{explain} logic,
  \item \textbf{remind} the reader of useful facts.
\end{itemize}
\end{mdframed}

\vspace{5mm}

On the next pages are two examples if pieces of mathematical work. Notice that understanding improves when words are included:

\vspace{15mm}

\begin{center}
\includestandalone[width=\textwidth]{understanding}
\end{center}

\vspace{15mm}

So the \fcolorbox{black}{orange!10}{words} are much more important than the \fcolorbox{black}{cyan!10}{mathematical notation}!

\newpage


\begin{mdframed}[linewidth=0.5mm, backgroundcolor=orange!10]
We should write words in such a way that it \textit{tells a story}.
\end{mdframed}

\vspace{5mm}

\begin{mdframed}[linewidth=0.5mm, backgroundcolor=cyan!10]
Mathematical notation should be used to \textit{give details}.
\end{mdframed}

\vspace{5mm}

The details might change, without changing the story. For example you may use a different mathematical method to find the same result.

In fact, sometimes these details are so unimportant to the story, we can ``outsource'' this work to a computer, that is \textit{use code}, allowing a computer to do those calculations for us.

\vspace{5mm}

\begin{mdframed}[linewidth=0.5mm, backgroundcolor=codebg]
When communicating work that uses code, we can replace the \fcolorbox{black}{cyan!10}{mathematical notation} with code.

\vspace{5mm}

(Even better, we might show both the \fcolorbox{black}{cyan!10}{mathematical notation} and code to communicate out all the details of the story told by the \fcolorbox{black}{orange!10}{words}.)
\end{mdframed}

\vspace{5mm}

In this case, the words are still the most important part!

\vspace{15mm}

\begin{center}
\includestandalone[width=\textwidth]{understanding_code}
\end{center}

\vspace{15mm}

But we still need to write code in such a way that we can read and understand what is going on. Code is not mathematical notation, and so we can write it more freely, and use descriptive variable names.

\newpage

\begin{center}
\LARGE{\textit{Writing Mathematics - Tin Can Example}}
\end{center}

\vspace{15mm}

\textbf{\textit{QUESTION:}}

\textbf{Consider a cylindrical tin can with radius $r$ and height $h$. Let its volume $V$ be non-zero and fixed. Find the relationship between $h$ and $r$ such that the surface area of the tin can is minimised.}

\vspace{5mm}
\hrule
\vspace{5mm}

\textbf{\textit{SOLUTION:}}

\vspace{5mm}

\begin{mdframed}[linewidth=0.5mm, backgroundcolor=orange!10]
First, let $r$ be the radius of the cylinder, $h$ be its height, and $V$ be its volume. We have expressions for the tin's volume and surface area:
\end{mdframed}

\begin{minipage}[t]{.37\textwidth} %
\begin{mdframed}[linewidth=0.5mm, backgroundcolor=cyan!10]
\begin{align*}
V &= \pi r^2 h \\
S &= 2\pi r^2 + 2 \pi rh
\end{align*}
\vspace{1.8cm}
\end{mdframed}
\end{minipage} %
\begin{minipage}[t]{.63\textwidth} %
\begin{minted}{python}
>>> import sympy as sym
>>> r = sym.Symbol("r")
>>> h = sym.Symbol("h")
>>> V = sym.Symbol("V")
>>> can_lid = sym.pi * r ** 2
>>> can_body = 2 * sym.pi * r * h

>>> surface_area = (2 * can_lid) + can_body
>>> volume = can_lid * h
\end{minted}
\end{minipage}

\begin{mdframed}[linewidth=0.5mm, backgroundcolor=orange!10]
As the volume $V$ is fixed, then we have a relationship between $r$ and $h$ that is dependant on $V$:
\end{mdframed}

\begin{minipage}[t]{.37\textwidth} %
\begin{mdframed}[linewidth=0.5mm, backgroundcolor=cyan!10]
\begin{align*}
V &= \pi r^2 h \\
\frac{V}{\pi r^2} &= h \\
\end{align*}
\vspace{0.8cm}
\end{mdframed}
\end{minipage} %
\begin{minipage}[t]{.63\textwidth} %
\begin{minted}{python}
>>> volume_equation = sym.Eq(volume, V)
>>> volume_equation


>>> sym.solveset(volume_equation, h)



>>> fixed_h = V / (sym.pi * (r ** 2))
\end{minted}
\vspace{-4.125cm}

\hspace{0.5cm}\begin{minipage}{3cm}$\pi h r^2 = V$\end{minipage}

\vspace{1cm}

\hspace{0.5cm}\begin{minipage}{3cm}$\displaystyle{\left\{\frac{V}{\pi r^2}\right\}}$\end{minipage}
\end{minipage}

\vspace{0.5cm}

\begin{mdframed}[linewidth=0.5mm, backgroundcolor=orange!10]
Therefore the surface area now becomes a function of $V$ and $r$:
\end{mdframed}

\begin{minipage}[t]{.37\textwidth} %
\begin{mdframed}[linewidth=0.5mm, backgroundcolor=cyan!10]
\begin{align*}
S &= 2\pi r^2 + 2 \pi rh \\
&= 2\pi r^2 + 2 \pi r\left(\frac{V}{\pi r^2}\right) \\
&= 2\pi r^2 + \frac{2 V}{r} \\
\end{align*}
\end{mdframed}
\end{minipage} %
\begin{minipage}[t]{.63\textwidth} %
\begin{minted}{python}
>>> surface_area = surface_area.subs({h: fixed_h})
>>> surface_area






\end{minted}
\vspace{-3.5cm}

\hspace{0.5cm}\begin{minipage}{3cm}$\displaystyle{\frac{2V}{r} + 2\pi r^2}$\end{minipage}
\end{minipage}
\vspace{0.5cm}

\begin{mdframed}[linewidth=0.5mm, backgroundcolor=orange!10]
The surface area is minimised when its derivative is equal to zero:
\end{mdframed}

\begin{minipage}[t]{.37\textwidth} %
\begin{mdframed}[linewidth=0.5mm, backgroundcolor=cyan!10]
\begin{align*}
0 &= \frac{dS}{dr} \\
&= 4\pi r - 2Vr^{-2}
\end{align*}
\vspace{1mm}
\end{mdframed}
\end{minipage} %
\begin{minipage}[t]{.63\textwidth} %
\begin{minted}{python}
>>> diff_surface = sym.diff(surface_area, r)
>>> differential_equation = sym.Eq(diff_surface, 0)
>>> differential_equation



\end{minted}
\vspace{-2cm}

\hspace{0.5cm}\begin{minipage}{3cm}$\displaystyle{-\frac{2V}{r^2} + 4\pi r = 0}$\end{minipage}
\end{minipage}
\vspace{0.5cm}

\begin{mdframed}[linewidth=0.5mm, backgroundcolor=orange!10]
Then solving gives $\tilde{r}$, the value of $r$ that minimises the surface area, in terms of $V$:
\end{mdframed}

\begin{minipage}[t]{.37\textwidth} %
\begin{mdframed}[linewidth=0.5mm, backgroundcolor=cyan!10]
\begin{align*}
4\pi \tilde{r} - \frac{2V}{\tilde{r}^{2}} &= 0 \\
4\pi \tilde{r} &= \frac{2V}{\tilde{r}^{2}} \\
4\pi \tilde{r}^3 &= 2V \\
\tilde{r}^3 &= \frac{V}{2\pi} \\
\tilde{r} &= \sqrt[3]{\frac{V}{2\pi}} \\
\end{align*}
\end{mdframed}
\end{minipage} %
\begin{minipage}[t]{.63\textwidth} %
\begin{minted}{python}
>>> sym.solveset(differential_equation, r)




>>> # Take the real valued solution
>>> r_tilde =(2 * V / sym.pi) ** (1 / sym.S(3))
>>> r_tilde_eq = sym.Eq(r_tilde, r)
>>> r_tilde_eq



\end{minted}
\vspace{-5.5cm}

\hspace{0.5cm}\begin{minipage}{\textwidth}$\left\{\frac{2^{\frac{2}{3}} \sqrt[3]{V}}{2 \sqrt[3]{\pi}}, - \frac{2^{\frac{2}{3}} \sqrt[3]{V}}{4 \sqrt[3]{\pi}} - \frac{2^{\frac{2}{3}} \sqrt{3} i \sqrt[3]{V}}{4 \sqrt[3]{\pi}}, - \frac{2^{\frac{2}{3}} \sqrt[3]{V}}{4 \sqrt[3]{\pi}} + \frac{2^{\frac{2}{3}} \sqrt{3} i \sqrt[3]{V}}{4 \sqrt[3]{\pi}}\right\} \setminus \left\{0\right\}$\end{minipage}

\vspace{2.7cm}

\hspace{0.5cm}\begin{minipage}{\textwidth}$r = \frac{\sqrt[3]{2} \sqrt[3]{V}}{\sqrt[3]{\pi}}$\end{minipage}
\end{minipage}

\vspace{0.5cm}

\begin{mdframed}[linewidth=0.5mm, backgroundcolor=orange!10]
However, as mentioned above, the volume $V$ is fixed, and is itself a function of $r$ and $h$. Substituting this in gives an implicit relationship between $\tilde{r}$ and $h$:
\end{mdframed}

\begin{minipage}[t]{.37\textwidth} %
\begin{mdframed}[linewidth=0.5mm, backgroundcolor=cyan!10]
\begin{align*}
\tilde{r} &= \sqrt[3]{\frac{V}{2\pi}} \\
&= \sqrt[3]{\frac{\pi \tilde{r}^2 h}{2\pi}} \\
&= \sqrt[3]{\frac{\tilde{r}^2 h}{2}}
\end{align*}
\vspace{3mm}
\end{mdframed}
\end{minipage} %
\begin{minipage}[t]{.63\textwidth} %
\begin{minted}{python}
>>> r_tilde_eq = r_tilde_eq.subs({V: volume})
>>> r_tilde_eq








\end{minted}
\vspace{-4.5cm}

\hspace{0.5cm}\begin{minipage}{\textwidth}$r = \frac{2^{\frac{2}{3}} \sqrt[3]{h r^2}}{2}$\end{minipage}
\end{minipage}

\vspace{0.5cm}

\begin{mdframed}[linewidth=0.5mm, backgroundcolor=orange!10]
Solving for either $\tilde{r}$ or $h$ will give an explicit relationship between $\tilde{r}$ and $h$, as required.
\end{mdframed}

\begin{minipage}[t]{.37\textwidth} %
\begin{mdframed}[linewidth=0.5mm, backgroundcolor=cyan!10]
\begin{align*}
\tilde{r} &= \sqrt[3]{\frac{\tilde{r}^2 h}{2}} \\
\tilde{r}^3 &= \frac{\tilde{r}^2 h}{2} \\
2\tilde{r}^3 &= \tilde{r}^2 h \\
2\tilde{r} &= h \\
\end{align*}
\end{mdframed}
\end{minipage} %
\begin{minipage}[t]{.63\textwidth} %
\begin{minted}{python}
>>> sym.solveset(r_tilde_eq, h)








\end{minted}
\vspace{-4.8cm}

\hspace{0.5cm}\begin{minipage}{\textwidth}$\left\{2r\right\}$\end{minipage}
\end{minipage}

\vspace{0.5cm}

\begin{mdframed}[linewidth=0.5mm, backgroundcolor=orange!10]
Therefore, for a fixed non-zero volume cylinder, the relationship between the radius $r$ and height $h$ that minimised the surface area is $h = 2r$.
\end{mdframed}




\newpage

\begin{center}
\LARGE{\textit{Writing Mathematics - Sales Example}}
\end{center}

\vspace{15mm}

\textbf{\textit{QUESTION:}}

\textbf{You have a new product that you would like to put on sale. It costs £8064 to set up a factory to begin manufacturing the products, and each product costs £2 to make. You expect the number of sales $S$ to be dependent on the price you set each unit, $P$, according to the relationship $S = 700 - 2P$. What price should be set to maximise profit, and how many units are sold?}

\vspace{5mm}
\hrule
\vspace{5mm}

\textbf{\textit{SOLUTION:}}

\vspace{5mm}

\begin{mdframed}[linewidth=0.5mm, backgroundcolor=orange!10]
From the information given, we can find expressions for the number of units sold, and the total profit:
\end{mdframed}

\begin{minipage}[t]{.6\textwidth} %
\begin{mdframed}[linewidth=0.5mm, backgroundcolor=cyan!10]
\vspace{-3mm}
\small{%
\begin{align*}
S &= 700 - 2P \\[2mm]
C &= 8064 + 2S \\
&= 8064 + 2(700 - 2P)\\
&= 9464 - 4P \\[2mm]
I &= PS \\
&= P(700 - 2P) \\
&= -2P^2 + 700P \\[2mm]
T &= I - C \\
&= (-2P^2 + 700P) - (9464 - 4P) \\
&= -2P^2 + 704P - 9464
\end{align*}%
}
\end{mdframed}
\end{minipage} %
\begin{minipage}[t]{.4\textwidth} %
\begin{minted}{python}
>>> import sympy as sym
>>> P = sym.Symbol("P")
>>> setup = 8064
>>> sales = 700 - (2 * P)
>>> costs = setup + (2 * sales)
>>> income = sales * P
>>> profit = income - costs
>>> sym.expand(profit)






\end{minted}
\vspace{-3.7cm}

\hspace{0.5cm}\begin{minipage}{\textwidth}$-2P^{2} + 704P - 9464$\end{minipage}
\end{minipage}

\vspace{0.5cm}

\begin{mdframed}[linewidth=0.5mm, backgroundcolor=orange!10]
We can see that the profit is a quadratic function of the price. This is a quadratic whose $x^2$ coefficient is negative, and so has a maximum. the maximum will lie halfway between its roots.

Therefore we find its roots:
\end{mdframed}

\begin{minipage}[t]{.3\textwidth} %
\begin{mdframed}[linewidth=0.5mm, backgroundcolor=cyan!10]
\vspace{-3mm}
\small{%
\begin{align*}
T &= -2P^2 + 704P - 9464 \\
&= -(2P + a)(P + b)
\end{align*}
where
\begin{align*}
2b + 1 &= -704 \\
ab &= 9464
\end{align*}
implies that:
\begin{align*}
a &= -28 \\
b &= -338
\end{align*}
and so:
\begin{equation*}
T = -(2P - 28)(P-338)
\end{equation*}
and so:
\begin{align*}
P_1 &= 14 \\
P_2 &= 338
\end{align*}
\vspace{-0.55cm}
}
\end{mdframed}
\end{minipage} %
\begin{minipage}[t]{.36\textwidth} %
\begin{mdframed}[linewidth=0.5mm, backgroundcolor=cyan!10]
\vspace{-3mm}
\small{%
\begin{equation*}
T = -2P^2 + 704P - 9464
\end{equation*}
using the quadratic equation:
\begin{align*}
P &= \frac{704 \pm \sqrt{(704^2 - 4(2)(9464))}}{2 \times 2)} \\
&= \frac{704 \pm \sqrt{419904}}{4} \\
&= 176 \pm 162
\end{align*}
\begin{align*}
P_1 &= 14 \\
P_2 &= 338
\end{align*}
\vspace{3cm}
}
\end{mdframed}
\end{minipage} %
\begin{minipage}[t]{.32\textwidth} %
\begin{minted}{python}
>>> sym.solveset(profit)




















\end{minted}
\vspace{-10.4cm}

\hspace{0.5cm}\begin{minipage}{\textwidth}$\left\{14, 3380\right\}$\end{minipage}
\end{minipage}

\vspace{0.5cm}

\begin{mdframed}[linewidth=0.5mm, backgroundcolor=orange!10]
The maximum of the quadratic function will be at the midpoint between the roots:
\end{mdframed}

\begin{minipage}[t]{.6\textwidth} %
\begin{mdframed}[linewidth=0.5mm, backgroundcolor=cyan!10]
\begin{align*}
\tilde{P} &= \frac{P_2 - P_1}{2} \\
&= \frac{388 - 14}{2} \\
&= 162
\end{align*}
\end{mdframed}
\end{minipage} %
\begin{minipage}[t]{.4\textwidth} %
\begin{minted}{python}
>>> P_tilde = (338 - 14) / 2
>>> P_tilde
162.0



\end{minted}
\end{minipage}

\begin{mdframed}[linewidth=0.5mm, backgroundcolor=orange!10]
Therefore the price that maximises the total profit is £162. We can substitute this into the expression for the sales and for the profit to find the maximum number of sales and maximum profit that can be made:
\end{mdframed}

\begin{minipage}[t]{.6\textwidth} %
\begin{mdframed}[linewidth=0.5mm, backgroundcolor=cyan!10]
\vspace{-0.6cm}
\begin{align*}
\tilde{S} &= 700 - 2\tilde{P} \\
&= 700 - 2(162) \\
&= 376 \\[3mm]
\tilde{T} &= -2\tilde{P}^2 + 704\tilde{P} - 9464 \\
&= -2(162^2) + (704 \times 162) - 9464 \\
&= 52096
\end{align*}
\vspace{-0.85cm}
\end{mdframed}
\end{minipage} %
\begin{minipage}[t]{.4\textwidth} %
\begin{minted}{python}
>>> sales.subs({P: midpoint})


>>> profit.subs({P: midpoint})




\end{minted}
\vspace{-3.8cm}

\hspace{0.5cm}\begin{minipage}{\textwidth}$376.0$\end{minipage} 

\vspace{1cm}

\hspace{0.5cm}\begin{minipage}{\textwidth}$52096.0$\end{minipage}
\end{minipage}

\vspace{0.5cm}

\begin{mdframed}[linewidth=0.5mm, backgroundcolor=orange!10]
Therefore, the maximum amount of profit that can be made is £52,096, by selling 376 units at a price of £162 each.
\end{mdframed}

\end{document}